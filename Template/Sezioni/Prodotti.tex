%----------------------------------------------------------------------------------------
%	DESCRIPTION OF THE PRODUCTS THAT ARE BEING EXPECTED FROM THE STAGE
%----------------------------------------------------------------------------------------
\section*{Prodotti attesi}
% Personalizzare definendo i prodotti attesi (facoltativo)
Lo stage sarà diviso in tre parti: una prima parte di formazione, una parte di preprocessing e una parte di implementazione, training e ottimizzazione della rete neurale. Per ognuno di questi punti è previsto almeno un prodotto in output.\\
Di seguito la lista completa:
\begin{enumerate}
    \item \textbf{Primo periodo: formazione} \\
		Lo studente deve produrre un report sul funzionamento delle reti neurali LSTM e sulla libreria keras\footnote{\href{https://keras.io/}{https://keras.io/} \\} di python utlizzata con tensorflow come backend. Lo scopo di questo report è dimostrare al docente proponente di aver compreso 
		il funzionamento delle reti neurali del tipo che lo studente stesso dovrà implementare e tutto quello che è necessario conoscere della libreria che verrà usata nella realizzazione della rete stessa. 
    
    \item \textbf{Secondo periodo: preprocessing} \\
		Lo studente dovrà produrre, oltre al dataset preprocessato, un report che spieghi nei dettagli il preproessing effettuato. Lo scopo è la creazione di un dataset che sia utilizzabile con la rete neurale e un prodotto 
		documentale che dimostri al professore proponente che lo studente ha lavorato correttamente e ha eseguito tutte le operazioni necessarie. Il dataset sarà costituito da partite di 50 giocatori differenti.
    
    \item \textbf{Terzo periodo: implementazione, training e ottimizzazione}\\	
		Lo studente nel terzo periodo dovrà realizzare la rete neurale vera e propria. Oltre a implementare la rete (basandosi sul paper 
		\href{https://link.springer.com/chapter/10.1007/978-3-030-62974-8_22}{PvP: Profiling Versus Player! Exploiting Gaming Data for Player Recognition})
		lo studente dovrà allenarla e ottimizzarne gli iperparametri. Il professore proponente si aspetta, inoltre, un report sul modello finale implementato 
		dallo studente che comprenda informazioni sugli iperparametri utilizzati e sulle performance.
    
\end{enumerate}

%Nel qual caso in cui lo studente, in seguito all'analisi, abbia ancora tempo a sua disposizione ... .

 
