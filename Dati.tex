%----------------------------------------------------------------------------------------
%   USEFUL COMMANDS
%----------------------------------------------------------------------------------------

\newcommand{\dipartimento}{Dipartimento di Matematica ``Tullio Levi-Civita''}

%----------------------------------------------------------------------------------------
% 	USER DATA
%----------------------------------------------------------------------------------------

% Data di approvazione del piano da parte del tutor interno; nel formato GG Mese AAAA
% compilare inserendo al posto di GG 2 cifre per il giorno, e al posto di 
% AAAA 4 cifre per l'anno
\newcommand{\dataApprovazione}{Data}

% Dati dello Studente
\newcommand{\nomeStudente}{Samuel}
\newcommand{\cognomeStudente}{Kostadinov}
\newcommand{\matricolaStudente}{1187605}
\newcommand{\emailStudente}{samuel.kostadinov@studenti.unipd.it}
\newcommand{\telStudente}{+ 39 393 80 54 308}

% Dati del Tutor Aziendale
\newcommand{\titoloTutorAziendale}{Prof.} 
\newcommand{\nomeTutorAziendale}{Mauro}
\newcommand{\cognomeTutorAziendale}{Conti}
\newcommand{\emailTutorAziendale}{mauro.conti@unipd.it}
\newcommand{\telTutorAziendale}{}
\newcommand{\ruoloTutorAziendale}{}

% Dati dell'Azienda
\newcommand{\ragioneSocAzienda}{Università degli studi di Padova}
\newcommand{\indirizzoAzienda}{Via VIII Febbraio 2, Padova (PD)}
\newcommand{\sitoAzienda}{https://www.unipd.it/}
%\newcommand{\emailAzienda}{mail@mail.it}
%\newcommand{\partitaIVAAzienda}{P.IVA 12345678999}

% Dati del Tutor Interno (Docente)
\newcommand{\titoloTutorInterno}{Prof.}
\newcommand{\nomeTutorInterno}{Davide}
\newcommand{\cognomeTutorInterno}{Bresolin}
\newcommand{\emailTutorInterno}{davide.bresolin@unipd.it}

\newcommand{\prospettoSettimanale}{
     % Personalizzare indicando in lista, i vari task settimana per settimana
     % sostituire a XX il totale ore della settimana
    \begin{itemize}
        \item \textbf{Prima Settimana - Studio individuale (40 ore)}
        \begin{itemize}
            \item Studio paper \href{https://link.springer.com/chapter/10.1007/978-3-030-62974-8_22}{PvP: Profiling Versus Player! Exploiting Gaming Data for Player Recognition};
            \item Studio individuale reti neurali LSTM;
			\item Studio contesti di applicazione di reti neurali LSTM.
        \end{itemize}
        \item \textbf{Seconda Settimana - Studio individuale (40 ore)} 
        \begin{itemize}
            \item Studio individuale libreria keras;
			\item Studio di esempi di reti neurali che utlizzano la libreria keras.
        \end{itemize}
        \item \textbf{Terza Settimana - Preprocessing: aggregazione (40 ore)} 
        \begin{itemize}
            \item Aggregazione dati e taglio sequenze (si manterranno 10 minuti di partita invece della partita completa);
        \end{itemize}
        \item \textbf{Quarta Settimana - Preprocessing: Standardizzazione (40 ore)} 
        \begin{itemize}
            \item Standardizzazione dei dati.
        \end{itemize}
        \item \textbf{Quinta Settimana - Preprocessing: Preparazione dataset (40 ore)} 
        \begin{itemize}
            \item Preparazione del dataset per essere letto dalla rete neurale;
			\item Creazione label.
        \end{itemize}
        \item \textbf{Sesta Settimana - Implementazione (40 ore)} 
        \begin{itemize}
            \item Implementazione di un primo modello di rete neurale;
			\item Training del modello.
        \end{itemize}
        \item \textbf{Settima Settimana - Feature selection (40 ore)} 
        \begin{itemize}
            \item Eliminazione feature non necessarie;
			\item Training del modello.
        \end{itemize}
        \item \textbf{Ottava Settimana - Model selection (40 ore)} 
        \begin{itemize}
            \item Selezione del modello con migliori prestazioni;
        \end{itemize}
    \end{itemize}
}

% Indicare il totale complessivo (deve essere compreso tra le 300 e le 320 ore)
\newcommand{\totaleOre}{320}

\newcommand{\obiettiviObbligatori}{
	 \item \underline{\textit{O01}}: Produzione report riguardante le reti neurali LSTM, il loro ambito di utilizzo e la libreria keras e le funzioni correlate alla libreria stessa;
	 \item \underline{\textit{O02}}: Preprocessing del dataset;
	 \item \underline{\textit{O03}}: Produzione report riguradante il preprocessing del dataset;
	 \item \underline{\textit{O04}}: Implementazione del modello di rete neurale;
	 \item \underline{\textit{O05}}: Produzione report riguradante il modello di rete neurale.
}

\newcommand{\obiettiviDesiderabili}{
	 \item \underline{\textit{D01}}: Studio della correlazione tra lunghezza delle sequenze di gioco e la precisione della predizione;
	 \item \underline{\textit{D02}}: Studio della correlazione tra la sequenza interessata della partita e la precisione della predizione;
	 \item \underline{\textit{D03}}: Studio dell'influenza delle feature generiche sulla precisione della predizione.
}

\newcommand{\obiettiviFacoltativi}{
	 \item \underline{\textit{F01}}: Studio dell'information loss dovuta all'operazione di aggregazione dei dati;
	 \item \underline{\textit{F02}}: Studio di diversi optimizer nell'allenamento della rete neurale.
}